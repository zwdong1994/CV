\documentclass[letterpaper,onecolumn,10pt]{article}
\pagestyle{empty}
\usepackage[hmargin=0.6in,vmargin=0.2in]{geometry}   %%Config the content location in the letters.


\begin{document}
\date{}
\title{\fontsize{20}{0}\selectfont \bf Weidong Zhu}
\author{\fontsize{12}{0}\selectfont\ Master, Xiamen University}
\maketitle
\thispagestyle{empty}
\section*{\fontsize{16}{0}\selectfont Personal Information \\ \rule[-7pt]{1\textwidth}{1pt}}
\textbf{Address: }No. 422 Siming South Road, 361005, Xiamen, China
\hfill
\textbf{Phone: }+86-18959218665\\
\raggedright
\textbf{My Github: }https://github.com/zwdong1994
\hfill
\textbf{Email: }zwdong1994@gmail.com\\
\raggedright
\textbf{Web: }https://zwdong1994.github.io/weidongzhu/


\section*{\fontsize{16}{0}\selectfont Education \\ \rule[-7pt]{1\textwidth}{1pt}}
2016 - now
\hfill
M.S., Computer Science and Technology, School of Information Science and Engineering\\
\raggedleft
{\bf Adviser:} Prof. Suzhen Wu and Prof. Bo Mao \\
{\bf Xiamen University, China}\\

2012 - 2016
\hfill
B.E., Information Security, School of Computer Science and Technology\\
\raggedleft
{\bf Huazhong University of Science and Technology, China}

\raggedright
\section*{\fontsize{16}{0}\selectfont Research Intersts \\ \rule[-7pt]{1\textwidth}{1pt}}
{\bf Areas:} Storage System, Parallel/Distributed Systems, Database\\
{\bf Focus}: Deduplication, SSD, K-V Stores, SSD-based Array

\section*{\fontsize{16}{0}\selectfont Conference Publications \\ \rule[-7pt]{1\textwidth}{1pt}}
\begin{itemize}
\item Bo Mao, \underline {Weidong Zhu}, Suzhen Wu and Hong Jiang. {\bf Double Blind Publication}. In Submission, 2019.
\end{itemize}

\begin{itemize}
\item Suzhen Wu, \underline {Weidong Zhu}, Guixin Liu, Hong Jiang, and Bo Mao. {\bf GC-aware Request Steering with Improved Performance and Reliability for SSD-based RAIDs}. In Proceedings of the 32nd IEEE International Parallel \& Distributed Processing Symposium (IPDPS�� 18). Vancouver, British Columbia, Canada, May 21-May 25, 2018.
\end{itemize}

\section*{\fontsize{16}{0}\selectfont Journal Publications \\ \rule[-7pt]{1\textwidth}{1pt}}
\begin{itemize}
\item Suzhen Wu, \underline {Weidong Zhu}, Hong Jiang, and Bo Mao. {\bf Double Blind Publication}. In Submission, 2019.
\end{itemize}

\begin{itemize}
\item Suzhen Wu, \underline {Weidong Zhu}, Bo Mao, Kuan-Ching Li. {\bf PP: Popularity-based Proactive Data Recovery for HDFS RAID systems}. Future Generation Computer Systems. 86: 1146-1153, September 2018.
\end{itemize}

\section*{\fontsize{16}{0}\selectfont Research \& Experience  \\ \rule[-7pt]{1\textwidth}{1pt}}
\raggedright
{\bf Research Assistant }
\hfill
{\bf Mentor: Suzhen Wu \hspace{0.3cm} July 2016 - now}\\
\raggedright
{\bf Advanced Storage Technology Lab, Xiamen University}
\begin{itemize}
\item \textbf{KV-based Stores} \\
CockroachDB is a distributed SQL database built on a transactional and strongly-consistent key-value store. And CockroachDB uses RocksDB as its key-value store engine. However, in the CockroachDB,  there are some CPU-consuming functions and operations like SELECT, encoding, are locate at the place where is far away from the storage level. So, the length of IO path is too long that will cause more writes and reads in order to transfer the data from the storage layer to the CockroachDB. Therefore, I modified the code of CockroachDB to move the CPU-consuming functions to the \underline {storage layer}(RocksDB), which is called CRDB-Near.
\end{itemize}
\begin{itemize}
\item \textbf{Deduplication} \\
Deduplication can significantly reduce write traffic and it has great profit for flash-based storage system, like SSD RAID. So, deduplication can help to improve the performance and reliability in the flash storage system. Moreover, with the advent of NVMe and 3D NAND flash technologies, the performance of flash-based storage systems has been improved significantly. Therefore, this brought a new problem that the hash operation might become the bottleneck in the deduplication systems. So, I proceeded my research with this direction. \hfill {\bf (in Submition)}
\end{itemize}
\begin{itemize}
\item \textbf{SSD-based Array} \\
SSD-based RAIDs suffer from significant performance degradation whenever user I/O requests conflict with the ongoing Garbage Collection (GC) operations which introduces tail latency. And it will also brought great tail latency in the SSD-based Array. Moreover, the uncorrectable error occurred of SSD in the SSD-based RAIDs will trigger the fail-recovery process. So, proposing a scheme that aware of the GC process within an SSD-based RAID, to address both the performance and reliability issues of SSD-based RAIDs alluded to above is our main target in this research project. \hfill {\bf (IPDPS'18)}
\end{itemize}

\raggedright
{\bf Join a Competition}
\hfill
{\bf 2014 - 2015}\\
{\bf Information Security Lab, Huazhong University of Science and Technology}

\begin{itemize}
\item \textbf{An USB-based Device with Encryption} \\
To ensure the security of USB-based device, encryption or the other security technology used on the USB-based device. However, existing schemes were always built in the software layer and its access control function is not perfect. So, we built a security system with USB-based device by using AES algorithm to encrypt the data and provided a strong access control system on the PC. Besides, the encryption system built on the firmware of the USB-based device.
\end{itemize}

\section*{\fontsize{16}{0}\selectfont Honors \& Awards  \\ \rule[-7pt]{1\textwidth}{1pt}}
IPDPS'18 Student Grant \hfill May 2018\\
Excellent Merit Student, Xiamen University, China \hfill Nov. 2017\\
National Scholarship, Ministry of Education of China \hfill Oct. 2017\\
Third Price of 2015 National College Students Information Security Contest, Ministry of Education of China\hfill Aug. 2015

\section*{\fontsize{16}{0}\selectfont Computer Skills  \\ \rule[-7pt]{1\textwidth}{1pt}}
\begin{tabbing}
   \hspace{2.3in}\= \hspace{2.6in}\= \kill % set up two tab positions
    {\bf Operating System:} \>Linux, Windows \\
    {\bf Database:} \>Leveldb, RocksDB, CockroachDB\\
    {\bf Programming:} \>C, C++, GO, Python, \LaTeX
\end{tabbing}
\end{document}
